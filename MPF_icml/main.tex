%%%%%%%% ICML 2025 SUBMISSION %%%%%%%%%%%%%%%%%

\documentclass{article}

% Recommended packages for figures and better typesetting:
\usepackage{microtype}
\usepackage{graphicx}
\usepackage{subfigure}
\usepackage{booktabs} % for professional tables
\usepackage{hyperref}
\usepackage{listings}  % For code listings
\usepackage{color}     % For colored text in listings
\usepackage{adjustbox} % For resizing tables to fit column width

% Configure the listings package for source code
\lstset{
  basicstyle=\ttfamily\small,
  breaklines=true,
  commentstyle=\color{gray},
  keywordstyle=\color{blue},
  stringstyle=\color{purple},
  numberstyle=\tiny\color{gray},
  frame=single,
  rulecolor=\color{black},
  backgroundcolor=\color{white},
  tabsize=2,
  captionpos=b
}

% Attempt to make hyperref and algorithmic work together better:
\newcommand{\theHalgorithm}{\arabic{algorithm}}

% Use the following line for the initial blind version submitted for review:
\usepackage[accepted]{icml2025}

\makeatletter
\def\Notice@String{Preliminary work. Under review by the International Conference
on Machine Learning (ICML). Do not distribute.}
\makeatother

% For theorems and such
\usepackage{amsmath}
\usepackage{amssymb}
\usepackage{mathtools}
\usepackage{amsthm}
\usepackage[capitalize,noabbrev]{cleveref}

% For drawing diagrams
\usepackage{tikz}
\usetikzlibrary{shapes.geometric, arrows, positioning, fit, backgrounds, matrix}

%%%%%%%%%%%%%%%%%%%%%%%%%%%%%%%%
% THEOREMS
%%%%%%%%%%%%%%%%%%%%%%%%%%%%%%%%
\theoremstyle{plain}
\newtheorem{theorem}{Theorem}[section]
\newtheorem{proposition}[theorem]{Proposition}
\newtheorem{lemma}[theorem]{Lemma}
\newtheorem{corollary}[theorem]{Corollary}
\theoremstyle{definition}
\newtheorem{definition}[theorem]{Definition}
\newtheorem{assumption}[theorem]{Assumption}
\theoremstyle{remark}
\newtheorem{remark}[theorem]{Remark}

% The \icmltitle you define below is probably too long as a header.
% Therefore, a short form for the running title is supplied here:
\icmltitlerunning{MPF: Aligning via Multi-Perspective Fusion}

\begin{document}

\twocolumn[
\icmltitle{MPF: Aligning and Debiasing Language Models via Post-Training Multi-Perspective Fusion}

% It is OKAY to include author information, even for blind
% submissions: the style file will automatically remove it for you
% unless you've provided the [accepted] option to the icml2025 package.
\icmlsetsymbol{equal}{*}

\begin{icmlauthorlist}
\icmlauthor{Your Name}{equal,aff1}
\icmlauthor{Co-Author}{equal,aff2}
\end{icmlauthorlist}

\icmlaffiliation{aff1}{Your Institution}
\icmlaffiliation{aff2}{Co-Author's Institution}

\icmlcorrespondingauthor{Your Name}{your.email@institution.edu}
\vskip 0.3in
% You may provide any keywords that you find helpful for describing your paper
\icmlkeywords{language models, bias mitigation, ensemble methods, multi-perspective learning, debiasing, model alignment}
]

\begin{abstract}
Multi-Perspective Fusion (MPF) is a novel post-training alignment framework for large language models (LLMs) developed in response to the growing need for easy bias mitigation. Built on top of the SAGED pipeline to interpret bias of LLM, MPF leverages multi-perspective generations to expose and align biases in LLM outputs with nuanced human-like baselines. By decomposing baseline outputs—such as sentiment distributions from HR professionals—into interpretable perspective components, MPF guides generation through sampling and balancing of responses, weighted by the probabilities obtained in the decomposition. Emperically, we evaluate MPF in the context of university and HR's standard, demonstrating its ability to align LLM sentiment distributions with expert baselines and reveal nuanced discorveries. Our results show that MPF offers a scalable and interpretable method for post-hoc alignment and bias mitigation, compatible with existing model architectures and requiring no extensive prompt engineering or fine-tuning.
\end{abstract}

% this must go after the closing bracket ] following \twocolumn[ ...

% This command actually creates the footnote in the first column
% listing the affiliations and the copyright notice.
\printAffiliationsAndNotice{\icmlEqualContribution} % use the standard text.

% Include each section
\section{Introduction}
\label{sec:introduction}

Recent advancements in large language models (LLMs) have highlighted both their capabilities for bias and their  harmful effect, raising significant concerns regarding alignment and fairness in deployed systems. Benchmarking frameworks such as BOLD and SAGED have emerged as key interpretability methods for uncovering model biases around specific concepts—such as gender (e.g., "female") or institutional affiliation (e.g., "X-university")—through the lens of examined linguistic features like sentiment, personality, or topical focus. 

In this paper, we introduce Multi-Perspective Fusion (MPF), a novel post-training alignment method that builds upon the interpretability capabilities of the SAGED pipeline. MPF offers an \textit{distributional  alignment} avoiding heavy prompt crafting or model fine-tuning—while remaining compatible with both. Via SAGED's automated construction of Question-Baseline (QB) benchmarks from dedicated texts, MPF can systematicaly compare between LLM outputs and implied human baselines in the texts, thereby mitigating bias through aligning LLM with the baselines' feature distribution. To do this, MPF re-composes the baseline's features into a weighted mixture of distribution from interpretable perspectives. Then, MPF uses these weights to probabilistically simulate LLM responses through sampling and aggregating, and leading to responses aligning with the baseline on particular features (i.e. sentiment), as shown in Fig~\ref{fig:enter-label}.

For our experiment, we instantiate MPF to capture HR sentiment toward different universities using a benchmark constructed with SAGED. As a baseline, we use LLMs role-playing Fortune 500 human resource (HR) professionals, leveraging their ability to simulate complex evaluation standards and reveal potential biases in LLM-based resume screening. To decompose this baseline, we define a set of sentiment-related perspectives—optimistic, realist, empathetic, cautious, and critical—and generate responses from these perspectives within the same benchmark. We then decompose the HR sentiment distribution across universities into a weighted combination of the sentiment distributions of these perspectives. For alignment, the resulting perspective weights are used as probabilities for stochastic routing, where each LLM generation is prompted using a perspective sampled according to its associated probability. To enhance the robustness and representativeness of the final output, we introduce an implicit voting ensemble strategy: multiple responses are generated through repeated stochastic routing and synthesized into a single coherent reply by a secondary aggregator LLM, which implicitly weights perspectives in proportion to their frequency of occurrence.

\begin{figure*}[ht]
    \centering
    \includegraphics[width=\textwidth]{img/MPF_demo.jpg}
    \caption{Enter Caption}
    \label{fig:enter-label}
\end{figure*}

The outcome of experiment demonstrates two notable findings: first, LLM HR fits human intuitions. LLM HR sentiment profiles align closely with critical perspectives when responding questions for lower-tier universities, but shift towards optimism when considering top-tier institutions. Second, through rigorous experiments and ablation studies, we find that applying MPF significantly reduces the sentiment discrepancy between baseline LLM HR and LLM outputs under MPF. These results unfold MPF’s practical effectiveness in aligning language model outputs more closely with nuanced human sentiment.




\subsection{Language Model Biases}
Language models have been shown to exhibit various forms of bias, including gender bias \cite{bolukbasi2016man}, racial bias, and other societal biases \cite{caliskan2017semantics}. These biases can manifest in different ways, from word embeddings to generated text.

\subsection{Debiasing Approaches}
Existing debiasing approaches can be broadly categorized into:

\subsubsection{Training-Based Methods}
\begin{itemize}
    \item Fine-tuning on debiased datasets
    \item Adversarial training
    \item Counterfactual data augmentation
\end{itemize}

\subsubsection{Post-Processing Methods}
\begin{itemize}
    \item Word embedding debiasing
    \item Output filtering
    \item Prompt engineering
\end{itemize}

\subsection{Ensemble Methods}
Ensemble methods have been used in various NLP tasks to improve performance and robustness. However, their application to bias mitigation has been limited. Our work extends ensemble approaches to specifically address bias in language models.

\subsection{Multi-Perspective Learning}
The concept of multiple perspectives has been explored in various domains, but its application to language model debiasing is novel. Our approach builds upon this concept while maintaining model coherence and performance. 
\section{Methodology}


Our MPE framework consists of two main components: the Mitigator and the ResponseGenerator. The Mitigator analyzes and optimizes perspective weights, while the ResponseGenerator applies these weights to generate debiased responses. 



\subsection{Distribution-Based Mitigation}
The Mitigator employs three distribution-based metrics to measure and reduce bias:

\begin{itemize}
    \item \textbf{Wasserstein Distance}: Measures the minimum cost of transforming one distribution into another
    \item \textbf{KL Divergence}: Quantifies the difference between probability distributions
    \item \textbf{Total Variation Distance}: Measures the maximum difference between probability distributions
\end{itemize}

\subsection{Bayesian Model Averaging}
We implement Bayesian Model Averaging (BMA) to combine different perspectives:

\begin{equation}
    P(y|x) = \sum_{i=1}^{n} w_i P_i(y|x)
\end{equation}

where $w_i$ are the optimized weights for each perspective $i$.

\subsection{Ensemble Generation}
The ResponseGenerator implements two approaches:

\begin{enumerate}
    \item \textbf{Routed Generation}: Selects a single perspective based on optimized weights
    \item \textbf{Ensemble Generation}: Combines multiple perspectives into a coherent response
\end{enumerate}

\subsection{Optimization Process}
The weight optimization process is formulated as:

\begin{equation}
    \min_w \sum_{i=1}^{n} w_i d(P_i, P_{target}) + \alpha \|w\|_2^2 + \beta \|w\|_1
\end{equation}

where:
\begin{itemize}
    \item $d$ is the chosen distribution metric
    \item $\alpha$ is the L2 regularization strength
    \item $\beta$ is the sparsity penalty strength
\end{itemize} 
\subsection{Experimental Setup}
We evaluate MPE on multiple benchmarks and compare it with state-of-the-art debiasing methods. Our experiments focus on:

\begin{itemize}
    \item Bias reduction effectiveness
    \item Model performance preservation
    \item Computational efficiency
    \item Generalization across different domains
\end{itemize}

\subsection{Datasets}
We use the following datasets for evaluation:

\begin{itemize}
    \item SAGED (Sentiment Analysis and Generation Evaluation Dataset)
    \item XNTION (Cross-Nation) benchmark
    \item Additional domain-specific datasets for generalization testing
\end{itemize}

\subsection{Baselines}
We compare MPE with several baselines:

\begin{itemize}
    \item Fine-tuning based debiasing methods
    \item Prompt engineering approaches
    \item Other ensemble-based methods
\end{itemize}

\subsection{Evaluation Metrics}
We use the following metrics to evaluate our approach:

\begin{itemize}
    \item Distribution-based metrics (Wasserstein, KL, TV)
    \item Task-specific performance metrics
    \item Bias measurement metrics
    \item Computational efficiency metrics
\end{itemize}

\subsection{Implementation Details}
Our implementation uses:

\begin{itemize}
    \item Python 3.8+
    \item PyTorch for model operations
    \item NumPy for numerical computations
    \item SciPy for optimization
\end{itemize}

The experiments were conducted on NVIDIA A100 GPUs with 80GB memory. 
\subsection{Bias Reduction Effectiveness}
Our experiments demonstrate that MPE significantly reduces bias across various metrics:

\begin{itemize}
    \item Average reduction in gender bias: X\%
    \item Average reduction in racial bias: Y\%
    \item Improvement in fairness metrics: Z\%
\end{itemize}

\subsection{Performance Preservation}
MPE maintains or improves model performance while reducing bias:

\begin{itemize}
    \item Task accuracy preservation: A\%
    \item Coherence score improvement: B\%
    \item Fluency metrics: C\%
\end{itemize}

\subsection{Computational Efficiency}
MPE's training-free nature provides significant advantages:

\begin{itemize}
    \item No additional training time required
    \item Minimal inference overhead
    \item Scalable to different model sizes
\end{itemize}

\subsection{Generalization Results}
MPE demonstrates strong generalization across:

\begin{itemize}
    \item Different domains
    \item Various model architectures
    \item Multiple languages
\end{itemize}

\subsection{Qualitative Analysis}
Case studies and examples demonstrate MPE's effectiveness in:

\begin{itemize}
    \item Reducing harmful biases
    \item Maintaining context awareness
    \item Preserving factual accuracy
\end{itemize} 
\subsection{Advantages of MPE}
Our approach offers several key advantages:

\begin{itemize}
    \item \textbf{Training-Free}: No additional training required, making it cost-effective and easily deployable
    \item \textbf{Flexibility}: Can be applied to any pre-trained language model
    \item \textbf{Interpretability}: The ensemble mechanism provides transparency in decision-making
    \item \textbf{Scalability}: Efficient implementation allows for real-time application
\end{itemize}

\subsection{Limitations}
Current limitations of MPE include:

\begin{itemize}
    \item Dependency on the quality of individual perspectives
    \item Potential trade-offs between bias reduction and performance
    \item Computational overhead in the ensemble process
\end{itemize}

\subsection{Future Work}
Potential directions for future research:

\begin{itemize}
    \item Integration with other debiasing techniques
    \item Extension to more complex bias types
    \item Development of automated perspective generation
    \item Application to other modalities
\end{itemize}

\subsection{Societal Impact}
The implications of our work include:

\begin{itemize}
    \item More fair and equitable AI systems
    \item Reduced propagation of harmful biases
    \item Better alignment with human values
    \item Improved trust in AI systems
\end{itemize} 
In this paper, we presented MPE, a novel training-free approach for debiasing and aligning language models through multi-perspective ensemble. Our method demonstrates significant improvements in reducing bias while maintaining model performance across various benchmarks.

Key contributions of our work include:
\begin{itemize}
    \item A novel training-free debiasing approach
    \item A sophisticated ensemble mechanism
    \item Comprehensive evaluation across multiple benchmarks
    \item Analysis of trade-offs and limitations
\end{itemize}

Our results show that MPE effectively reduces bias while preserving model performance, making it a practical solution for real-world applications. The training-free nature of our approach makes it particularly appealing for deployment in production environments.

Future work will focus on addressing current limitations and extending the approach to more complex bias types and other modalities. We believe that MPE represents a significant step forward in the development of fair and equitable AI systems. 

\bibliography{references}
\bibliographystyle{icml2025}

\end{document} 